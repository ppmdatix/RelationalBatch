\section{Deep Learning}\label{deepLearning}

% \begin{itemize}
%     \item Maxime \ok ? \no ?
%     \item Thierry \ok ? \no ?
%     \item Victor \ok ? \no ?
% \end{itemize}



Most deep architectures are built to support numerical data such as image or audio but \ohe approach has already been applied by \cite{RevisitingDeepForTabular} for example and is called \textit{Feature Tokenizer}. Applying \textbf{one-hot-encoding} to feed neural networks is theoretically equivalent to use a different network for each symbol. We illustrate that on a dense linear layer as depicted in Figures \ref{fig:catModelOneHot}. This use of different networks for each symbols raises the issue depicted in \ref{sec:motication} when used with \textit{batch} during training. Gradient estimation done by \textit{batch} implies that the parameters are used on every observation, which totally makes sense for classical Deep Learning (i.e. with no one-hot encoded data). If a symbol is not present in a \textit{batch}, it means that there is no accumulated gradient, which is different than accumulated gradient is zero.



\begin{figure*}
\centering
\begin{subfigure}{0.48\linewidth}
\centering
\begin{tikzpicture}[scale=0.8]
\begin{scope}[every node/.style={square,thick,draw,minimum size=0.8pt}]
\begin{footnotesize}
    \node (num1) at (0,8)  {};
    \node (num1bis) at (2,8)  {};
    \node (num2) at (0,5.5)   {};
    \node (num3) at (0,4)   {};
    \node (hidden1) [color=red] at (4,5)     {};
    \node (hidden2) [color=red] at (4,7)     {};
\end{footnotesize}
\end{scope}
\begin{scope}[every node/.style={square,thick,draw,minimum size=2.05cm}]
\begin{small}
    \node (container) [color=blue] at (1,8) {};
\end{small}
\end{scope}
\begin{scope}[>={Stealth[black]},
              every node/.style={fill=white,circle},
              every edge/.style={thin, color=black}]
    \path [->] (num1) edge[draw=black, dashed] (num1bis);
    \path [->] (num1bis) edge[draw=black] (hidden1);
    \path [->] (num1bis) edge[draw=black] (hidden2);
    \path [->] (num2) edge[draw=black, dashed] (hidden1);
    \path [->] (num2) edge[draw=black, dashed] (hidden2);
    \path [->] (num3) edge[draw=black, dashed] (hidden1);
    \path [->] (num3) edge[draw=black, dashed] (hidden2);
\end{scope}
\end{tikzpicture}
\caption{Dense Layer}
\label{fig:catModelOneHotPART1}
\end{subfigure}
\begin{subfigure}{0.48\linewidth}
\centering
\begin{tikzpicture}[scale=0.6]
\begin{scope}[every node/.style={square,thick,draw,minimum size=1.2pt}]
%%    \node (in) at (-1.2,6)  {};
    \node (hidden2) at (4,5) {};
    \node (hidden1) at (4,7) {};
\end{scope}

\begin{scope}[every node/.style={square,thick,draw,minimum size=25pt,fill=gray,opacity=.2,text opacity=1}]
\begin{footnotesize}
    \node (A) at (0,8)  {$is_{blue}$};
    \node (B) at (0,6)  {$is_{pink}$};
    \node (C) at (0,4)  {$is_{green}$};
\end{footnotesize}
\end{scope}

\begin{scope}[every node/.style={square,thick,draw,minimum size=4cm}]
    \node (container) [color=blue] at (1.7,6) {};
\end{scope}
\begin{scope}[every node/.style={square,thick,draw,minimum size=1.2pt}]
    \node (out1) [color=red] at (7,6) {};
    \node (out2) [color=red] at (7,2) {};
\end{scope}

\begin{scope}[>={Stealth[black]},
              every node/.style={fill=white,circle},
              %%every edge/.style={very thick, color=black}]
              every edge/.style={thin, color=black}]
    \path [->] (A) edge[draw=black, dashed] (hidden1);
    \path [->] (A) edge[draw=black, dashed] (hidden2);
    \path [->] (B) edge[draw=black, dashed] (hidden1);
    \path [->] (B) edge[draw=black, dashed] (hidden2);
    \path [->] (C) edge[draw=black, dashed] (hidden1);
    \path [->] (C) edge[draw=black, dashed] (hidden2);
    \path [->] (hidden1) edge[draw=black] (out1);
    \path [->] (hidden2) edge[draw=black] (out2);

\end{scope}
\end{tikzpicture}
\caption{\ohe of a dense layer}
\label{fig:catModelOneHotPART2}
\end{subfigure}
\begin{subfigure}{0.5\linewidth}
\centering
\begin{tikzpicture}[scale=0.6]
\begin{scope}[every node/.style={square,thick,draw,minimum size=1.2pt}]
%%    \node (in) at (-1.2,6)  {};
    \node (hidden2) at (4,5) {};
    \node (hidden1) at (4,7) {};
\end{scope}

\begin{scope}[every node/.style={square,thick,draw,minimum size=25pt,fill=gray,opacity=.2,text opacity=1}]
\begin{footnotesize}
    \node (A) at (0,8)  {$is_{blue}$};
    \node (B) at (0,6)  {$is_{pink}$};
    \node (C) at (0,4)  {$is_{green}$};
\end{footnotesize}
\end{scope}

\begin{scope}[every node/.style={square,thick,draw,minimum size=4cm}]
    \node (container) [color=blue] at (1.7,6) {};
\end{scope}
\begin{scope}[every node/.style={square,thick,draw,minimum size=1.2pt}]
    \node (out1) [color=red] at (7,6) {};
    \node (out2) [color=red] at (7,2) {};
\end{scope}

\begin{scope}[>={Stealth[black]},
              every node/.style={fill=white,circle},
              %%every edge/.style={very thick, color=black}]
              every edge/.style={thin, color=black}]
    \path [->] (A) edge[draw=black, dashed] (hidden1);
    \path [->] (A) edge[draw=black, dashed] (hidden2);
    \path [->] (B) edge[draw=black, dashed] (hidden1);
    \path [->] (B) edge[draw=black, dashed] (hidden2);
    \path [->] (C) edge[draw=red, dashed] (hidden1);
    \path [->] (C) edge[draw=red, dashed] (hidden2);
    \path [->] (hidden1) edge[draw=red] (out1);
    \path [->] (hidden2) edge[draw=red] (out2);

\end{scope}
\end{tikzpicture}
\caption{dense layer relative to she \textit{green} symbol}
\label{fig:catModelOneHotPART2}
\end{subfigure}

\caption{\ohe for neural networks}
\label{fig:catModelOneHot}
\end{figure*}






